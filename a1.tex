\documentclass{article}
\usepackage[english]{babel}
\usepackage{amssymb,amsmath}

\begin{document}
\title{Assignment 2}
\author{Ana Arrabal Ortiz}

\begin{paragraph} Variety expansion: labour or lab equipment as RyD input?
Compare the models in Section 3.2.1 (discussed in class) and Section 3.2.2 in AH, both in terms of assumptions and in terms of results.
\end{paragraph}
 
\subsection{a}
Write the equations of the two models and comment on the differences.
    Product variety model: 
    \begin{equation}
    Y_{t}= L^{1-\alpha}\int_{0}^{M_{t}} X_{i}^{\alpha} di    
    \end{equation}
 
      \begin{equation}
      x = L\alpha^{2/1-\alpha}  
       \end{equation}
     \begin{equation}
    \pi= ((1-\alpha)/\alpha) L\alpha^{2/(1-\alpha)} 
     \end{equation}
     \begin{equation}
  r = \rho+\epsilon g
  \end{equation}
  \begin{equation}
 g=\lambda_{PV}1-\alpha/\alpha L\alpha^{2/1-\alpha}-\rho 
 \end{equation}
 \begin{equation}
 M = S_{R}L\alpha^{\alpha/1-\alpha}\lambda_{PV} 
  \end{equation}
 

  \title{Romer model}
 
 \begin{equation} 
 Y_{t} = L_{1}^{1-\alpha} \int_{0}^{M{t}} X_{i}^{\alpha}di
 \end{equation}
 
   \begin{equation} 
    x = L_{1}\alpha^{2/1-\alpha}
    \end{equation}
    
    \begin{equation} 
    \pi = ((1-\alpha)/\alpha) L_{1}\alpha^{2/1-\alpha}
    \end{equation}
    
    \begin{equation} 
    r= \alpha(\lambda_{R} L-g)
     \end{equation}
     \begin{equation} 
     g=(\lambda_{R}\alpha L-\rho)/alpha+\epsilon
     \end{equation}
 
\begin{equation}
M = \lambda_{R} M_{t} L_{2}
\end{equation}

\title{Table's legend:}

- $Y_{t}$ : Final good production function
- $x$: Profit maximizing quantity
- $\pi$: Intermediate monopolist profits
- $r$: Research-arbitrage
- $g$: Growth rate
- $M$: Growth rate of variety

\underline {Assumptions of both Models}:
In the Simple Variant of the Product-Variety Model, for simplicity we suppose that no one has a demand for leisure time, so each person offers her one unit of labour for sale ineslastically. The utility in each period depends only on consumption, according to the following Euler equation:

\begin{equation} 
 u(c)= c^{1-\epsilon}/(1-\epsilon)
 \end{equation}

where $\epsilon>0$

Let $X_{t}$ be the total amount of final good used in producing intermediate products. According to the one-for-one technology,  $X_{t}$ must equal total intermediate output: 

\begin{equation} 
X_{t} = \int_{0}^{M_{t}} X_{i} di
 \end{equation}

For simplicity, we assume that each intermediate product is produced in the same amount x: 

$x = x_{i}$ for all i

\begin{paragraph}The main difference between both models is the following alternative assumption of the Romer Model with Labour as RyD Input: we now suppose that labour can be used either in manufacturing the final good  $L_{1}$ or alternatively in research $L_{2}$. Labour used in these two activities must add up to the total labour supply $(L)$, which we again assume to be a given constant:
 \end{paragraph}
 $L= L_{1}+ L_{2}$

Finally, in both models we assume free market entry.

\underline {Results of the Models:}
We can say that each intermediate good producer is a monopolist for this product; the monopolist seeks to maximize the flow of profit at each date, measured in units of final good:
$\pi$
		where pi is the price in units of final good.
Since the price of an input to a perfectly competitive industry (free market entry assumption) is the value of each marginal product, in the Simple Variant of the Product-Variety Model in equilibrium the profit flow will be:
 
\begin{equation} 
\pi= ((1-\alpha)/\alpha) L\alpha^{2/(1-\alpha)}
\end{equation}

Because of the maximization problem is the same in both models, in the Romer Model in equilibrium we obtain the following profit flow:
 
$\pi =((1-\alpha)/\alpha) L_{1}\alpha^{2/(1-\alpha)}$ 
 
These equations which are equal except from L1 instead of L, state that profits are in function of the Labour meaning that when L increases also profits increase.
If profits increase, the prospect of these rents will motivate research activities aimed at discovering new varieties.
Due to the assumption of $x=xi$  for all I then, $x=X_{t}/M_{t}$, using $M_{t}x$ which is the total cost of producing $x$, we can plug in into the production relation and see that the final good output and the economy?s GDP will both be proportional to the degree of product variety:
 
\begin{equation} 
Y_{t} = M_{t}(L^{1-\alpha}x^{\alpha}-x)
\end{equation}
 
Therefore, the growth rate of GDP will be the proportional growth rate of product variety:
 
 $g=M$ 
 
 Regarding the Simple Variant of the Product-Variety Model; using the research arbitrage equation and the equilibrium profit flow we obtain:
 
\begin{equation} 
g=(1/\epsilon) (\lambda (1-\alpha/\alpha L\alpha^{2/1-\alpha}-\rho)
\end{equation}

From this equation we can immediately see that growth increases with the productivity of research measured by parameter lambda (number of patents) and with the size of the economy measured by labour supply L, and decreases with the rate of time preference $\rho$.

 However, over time RyD became more and more difficult in fact:
		     where $\phi$ <1
In the Romer Model case we have:

 $g = M = \lambda L_{2}$	so we have $r=(L-g)$, so we obtain:

\begin{equation}
g = (\lambda\alpha L-\rho)/(\alpha+\epsilon)
\end{equation}

Thus, we obtain similar conclusions; in fact we can say that growth increases with: productivity of research activities (lambda) and the size of the economy (L). Furthermore, the preceding equilibrium growth rate is always less than the social optimum because: intermediate firms do not internalize their contribution to the division of labour and researchers do not internalize research spill overs. This is because ideas are non-rival and exclusive.
\end{document}
